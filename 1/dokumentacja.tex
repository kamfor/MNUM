% TeX encoding = utf8
% TeX spellcheck = pl_PL 
\documentclass[a4paper, 8pt]{article}
\usepackage[utf8]{inputenc}
\usepackage[polish]{babel}
\usepackage{polski}
\usepackage{graphicx}
\usepackage{listings}
\usepackage{amsfonts}
\usepackage{geometry}
\usepackage{multicol}
\usepackage{latexsym}
\usepackage{enumerate}
\usepackage{color} %red, green, blue, yellow, cyan, magenta, black, white
\definecolor{mygreen}{RGB}{28,172,0} % color values Red, Green, Blue
\definecolor{mylilas}{RGB}{170,55,241}

\author{Kamil Foryszewski}
\title{Dokumentacja projektu laboratoryjnego numer 1 przedmiot MNUM}
\frenchspacing

\newgeometry{tmargin=2cm, bmargin=2cm, lmargin=2cm, rmargin=2cm}
\pagestyle{empty}


\begin{document}

\lstset{language=Matlab,%
    basicstyle=\color{red},
    breaklines=true,%
    morekeywords={matlab2tikz},
    keywordstyle=\color{blue},%
    morekeywords=[2]{1}, keywordstyle=[2]{\color{black}},
    %identifierstyle=\color{black},%
    stringstyle=\color{mylilas},%
    commentstyle=\color{mygreen},%
    showstringspaces=false,
    numbers=left,%
    numberstyle={ \color{black}},% size of the numbers
    numbersep=8pt, % this defines how far the numbers are from the text
    emph=[1]{for,endfor,endwhile,endif,break},emphstyle=[1]\color{blue}, %some words to emphasise
    emph=[2]{,.}, emphstyle=[2]\color{yellow},%
}

\maketitle
\tableofcontents


\section{Epsilon maszynowy}
\subsection*{Polecenie}
Proszę napisać program wyznaczjący dokładność maszynową komputera i wyznaczyć ją na swoim komputerze.
\subsection*{Opis teoretyczny}
\indent

Epsilon maszynowy jest to maksymalny błąd względny reprezentacji zmiennoprzecinkowej. Zależy on jedynie od liczby bitów mantysy i nazywany jest dokładnością maszynową. Aby go wyznaczyć należy znajeźć najmniejszą nieujemną liczbę która, dodada do jedności daje wynik różny od jedności. Nie należy mylić go z dużo mniejszą liczbą nazywaną liczbą róźną od zera, którą wyznacza się w podobny sposób. Epsylon maszynowy jest zależny od liczby bitów mantysy w reprezentacji zmiennoprzecinkowej. Epsilon maszynowy możemy wyznaczyć przy pomocy następującego algorytmu podanego jako lista kroków:
\begin{enumerate}
  \item $a = 1, b = 2$
  \item dopóki $a$ różne od $1$ $eps = x/2 b = 1 + x$
  \item wyświetl eps
\end{enumerate} 
Poniżej kod programu wyznaczającego epsylon maszynowy w programie Matlab:


\subsection*{Realizacja w programie Matlab}

\begin{lstlisting}
a = 1; % MATLAB comment 
b = 2; 
c = a^2 + b^2;
for i=1:n
endfor
endwhile
endif
\end{lstlisting}






	
\end{document}


