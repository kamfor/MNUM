% TeX encoding = utf8
% TeX spellcheck = pl_PL 
\documentclass[a4paper, 11pt]{article}
\usepackage[utf8]{inputenc}
\usepackage[polish]{babel}
\usepackage{polski}
\usepackage{graphicx}
\usepackage{listings}
\usepackage{amsfonts}
\usepackage{geometry}
\usepackage{multicol}
\usepackage{latexsym}
\usepackage{enumerate}
\usepackage{hyperref}
\usepackage{wrapfig}
\usepackage{color} %red, green, blue, yellow, cyan, magenta, black, white
\definecolor{mygreen}{RGB}{28,172,0} % color values Red, Green, Blue
\definecolor{mylilas}{RGB}{170,55,241}

\author{Kamil Foryszewski}
\title{Dokumentacja projektu laboratoryjnego numer 3 przedmiot MNUM}
\frenchspacing

\newgeometry{tmargin=2cm, bmargin=2cm, lmargin=2cm, rmargin=2cm}
\pagestyle{empty}


\begin{document}

\lstset{language=Matlab,%
    basicstyle=\color{red},
    breaklines=true,%
    morekeywords={matlab2tikz},
    keywordstyle=\color{blue},%
    morekeywords=[2]{1}, keywordstyle=[2]{\color{black}},
    identifierstyle=\color{black},%
    stringstyle=\color{mylilas},%
    commentstyle=\color{mygreen},%
    showstringspaces=false,
    numbers=right,%
    numberstyle={ \color{black}},% size of the numbers
    numbersep=5pt, % this defines how far the numbers are from the text
    emph=[1]{for,endfor,endwhile,endfunction,endif,break},emphstyle=[1]\color{blue}, %some words to emphasise
    emph=[2]{,.}, emphstyle=[2]\color{yellow},%
}

\maketitle
\tableofcontents

\section{Zadanie 1}

\subsection{Polecenie}
Proszę znaleźć wszystkie zera funkcji\\
\\
{$f(x) = 1.4*sin(x)-e^{x}=6*x-0.5$}\\
\\
w przedziale $[-5,5]$, używając dla każdego każdego zera programu z implementacją \\
a) metody bisekcji\\
b) metody siecznych\\
c) metody Newtona\\
 

\subsection{Metoda bisekcji}
\subsubsection{Opis teoretyczny}
Teoretyczny zarys metody bisekcji możemy przybliżyć poniższym algorytmem:
\begin{enumerate}
  \item Począwszy od przedziału startowego $[a,b]$ = $[a_{0},b_{0}]$ obliczamy środek przedziału $c_{n}$\\
  $c_{n} = \frac{a_{n}+b_{n}}{2}$\\
  i obliczamy wartość $f(x)$ w tym punkcie. 
  \item Obliczamy iloczyny $f(a_{n})*f(c_{n})$ oraz $f(b_{n})*f(c_{n})$ I jako nowy przedział $[a_{n=1},b_{n+1}$
  wybieramy ardumenty tego iliczynu którego wartość jest ujemna. 
\end{enumerate} 
Kroki te powtarzamy aż do mementu uzyckania $f(c_{n})<\delta$ gdzie $\delta$ to oczekiwana dokładność rozwiązania. W przypadku "płaskich" funkcji warto też konrtolować długość rozpatrywanego przedziału. 
Dokładność wyniku zalezy jedynie od ilośći iteracji dlatego metoda jest zbierzna liniowo z ilorazem zbierzności 0,5. Co czyni ją stosunkowo wolno zbierzną w przypadku wyboru szerokiego przedziału początkowego. 



\subsubsection{Realizacja w programie Matlab}
\begin{lstlisting}

\end{lstlisting}

\subsection{Metoda siecznych}
\subsubsection{Opis teoretyczny}
Teoretyczny zarys metody siecznych możemy przybliżyć poniższym algorytmem:
\begin{enumerate}
  \item Począwszy od przedziału startowego $[a,b]$ = $[a_{0},b_{0}]$ obliczamy punkt $x_{n}$ jako miejcse przecięcia siecznej funkcji przechodzącej prze punkty $[a_{n},b_{n}]$ gdzie $x_{n}=$
  \item Nastpepnie nowy przedział oznaczamy $x_{n+1}=$ 
\end{enumerate} 
Kroki te powtarzamy aż do mementu uzyckania $f(c_{n})<\delta$ gdzie $\delta$ to oczekiwana dokładność rozwiązania. Rząd zbierzności metody siecznych wynosi $(1+sqrt(5))/2$ co jest w przybliżeniu równe $1.618$. 
Jwst więc ona dużo szybsza od metody bisekcji, jednak jest zbierzna jedynie lokalnie. Jeżeli ine zadbamy o wybór odpowiedniego przedziału początkowego może okazać się wogóle nie zbieżna. 


\subsubsection{Realizacja w programie Matlab}
\begin{lstlisting}

\end{lstlisting}

\subsection{Metoda Newtona}
\subsubsection{Opis teoretyczny}


\subsubsection{Realizacja w programie Matlab}
\begin{lstlisting}

\end{lstlisting}

\subsection{Analiza danych wejściowych}
W celu wyznaczenia przedziałów izolacji miejsc zerowych został wykorsyztany algorytm opisany w skrypcie prof. Tatjewskiego. Wstępna analiza danych rozpoczyna się od wygenerowania wykresu funkcji w danym przedziale i na tej podstawie wyboru przedziału startowego dla algorytmu. Nastepnie w podanycm przedziale w pętli badany jest znak iloczynu funkcji w punktach graniczynych. Jezeli jest on ujemny, oznacza to występowanie miejsca zerowego w danycm przedziale. Jeżeli nie to przedział jest rozszerzany do momentu przekroczenia przedziału danego w zadaniu. 
Poniżej wykres funkcji z zaznaczonymi przedziałami izolacji wyznaczonymi przez algorytm. 


\subsection{Skrypt generujący rozwiązanie zadania w programie Matlab}
\begin{lstlisting}

\end{lstlisting}
\vspace{1cm}


\subsection{Wyniki}


\subsection{Wnioski}


\section{Zadanie 2}

\subsection{Polecenie}
Używając metody Mullera MM2, proszę znaleźć wszystkie pierwiastki rzeczywiste i zespolone wielomianu \\
$f(x) = a_{4}x^4+a_{3}x^3+a_{2}x^2+a_{1}x+a_{0}$ 
$
\left[
\begin{array}{ccccc}
       a_{4} & a_{3} & a_{2} & a_{1} & a_{0}
\end{array}
\right]
=
\left[
\begin{array}{ccccc}
       1 & 2 & 4 & -1 & 8
\end{array}
\right]$


\subsection{Metoda Mullera MM2}


\subsection{Realizacja w programie Matlab}
\begin{lstlisting}

\end{lstlisting}

\vspace{2cm}

\subsection{Wyniki działania programu}

\vspace{10cm}
\subsection{Wnioski}

	
\end{document}


