% TeX encoding = utf8
% TeX spellcheck = pl_PL 
\documentclass[a4paper, 11pt]{article}
\usepackage[utf8]{inputenc}
\usepackage[polish]{babel}
\usepackage{polski}
\usepackage{graphicx}
\usepackage{listings}
\usepackage{amsfonts}
\usepackage{geometry}
\usepackage{multicol}
\usepackage{latexsym}
\usepackage{enumerate}
\usepackage{hyperref}
\usepackage{wrapfig}
\usepackage{color} %red, green, blue, yellow, cyan, magenta, black, white
\definecolor{mygreen}{RGB}{28,172,0} % color values Red, Green, Blue
\definecolor{mylilas}{RGB}{170,55,241}

\author{Kamil Foryszewski}
\title{Dokumentacja projektu laboratoryjnego numer 3 przedmiot MNUM}
\frenchspacing

\newgeometry{tmargin=2cm, bmargin=2cm, lmargin=2cm, rmargin=2cm}
\pagestyle{empty}


\begin{document}

\lstset{language=Matlab,%
    basicstyle=\color{red},
    breaklines=true,%
    morekeywords={matlab2tikz},
    keywordstyle=\color{blue},%
    morekeywords=[2]{1}, keywordstyle=[2]{\color{black}},
    identifierstyle=\color{black},%
    stringstyle=\color{mylilas},%
    commentstyle=\color{mygreen},%
    showstringspaces=false,
    numbers=right,%
    numberstyle={ \color{black}},% size of the numbers
    numbersep=5pt, % this defines how far the numbers are from the text
    emph=[1]{for,endfor,endwhile,endfunction,endif,break},emphstyle=[1]\color{blue}, %some words to emphasise
    emph=[2]{,.}, emphstyle=[2]\color{yellow},%
}

\maketitle
\tableofcontents

\section{Zadanie 1}

\subsection{Polecenie}
 

\subsection{Metoda bisekcji}
\subsubsection{Opis teoretyczny}


\subsubsection{Realizacja w programie Matlab}
\begin{lstlisting}

\end{lstlisting}

\subsection{Metoda siecznych}
\subsubsection{Opis teoretyczny}


\subsubsection{Realizacja w programie Matlab}
\begin{lstlisting}

\end{lstlisting}

\subsection{Metoda Newtona}
\subsubsection{Opis teoretyczny}


\subsubsection{Realizacja w programie Matlab}
\begin{lstlisting}

\end{lstlisting}


\subsection{Skrypt generujący rozwiązanie zadania w programie Matlab}
\begin{lstlisting}

\end{lstlisting}
\vspace{1cm}


\subsection{Wyniki}


\subsection{Wnioski}


\section{Zadanie 2}

\subsection{Polecenie}

\subsection{Metoda Mullera MN2}


\subsection{Realizacja w programie Matlab}
\begin{lstlisting}

\end{lstlisting}

\vspace{2cm}

\subsection{Wyniki działania programu}

\vspace{10cm}
\subsection{Wnioski}

	
\end{document}


