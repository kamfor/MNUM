% TeX encoding = utf8
% TeX spellcheck = pl_PL 
\documentclass[a4paper, 11pt]{article}
\usepackage[utf8]{inputenc}
\usepackage[polish]{babel}
\usepackage{polski}
\usepackage{float}
\usepackage{graphicx}
\usepackage{listings}
\usepackage{amsfonts}
\usepackage{geometry}
\usepackage{multicol}
\usepackage{latexsym}
\usepackage{enumerate}
\usepackage{hyperref}
\usepackage{wrapfig}
\usepackage{color} %red, green, blue, yellow, cyan, magenta, black, white
\definecolor{mygreen}{RGB}{28,172,0} % color values Red, Green, Blue
\definecolor{mylilas}{RGB}{170,55,241}

\author{Kamil Foryszewski}
\title{Dokumentacja projektu laboratoryjnego numer 4 przedmiot MNUM}
\frenchspacing

\newgeometry{tmargin=2cm, bmargin=2cm, lmargin=2cm, rmargin=2cm}
\pagestyle{empty}


\begin{document}

\lstset{language=Matlab,%
    basicstyle=\color{red},
    breaklines=true,%
    morekeywords={matlab2tikz},
    keywordstyle=\color{blue},%
    morekeywords=[2]{1}, keywordstyle=[2]{\color{black}},
    identifierstyle=\color{black},%
    stringstyle=\color{mylilas},%
    commentstyle=\color{mygreen},%
    showstringspaces=false,
    numbers=right,%
    numberstyle={ \color{black}},% size of the numbers
    numbersep=5pt, % this defines how far the numbers are from the text
    emph=[1]{for,endfor,endwhile,endfunction,endif,break},emphstyle=[1]\color{blue}, %some words to emphasise
    emph=[2]{,.}, emphstyle=[2]\color{yellow},%
}

\maketitle
\tableofcontents

\section{Wstęp}
\subsection{Polecenie}
Ruch punktu jest opisany równaniami:
$$ x_{1}' = x_{2} + x_{1}(0,04 - x_{1}^{2} - x_{2}^{2}) $$
$$ x_{2}' = -x_{1} + x_{2}(0,04 - x_{1}^{2} - x_{2}^{2}) $$
Należy obliczyć przebieg trajektorii na przedziale [0.20] dla następujących warunków początkowych: \\
a) $ x_{1}(0) = 9 $ \hspace{1cm} $ x_{2}(0) = 8 $ \\
b) $ x_{1}(0) = 0 $ \hspace{1cm} $ x_{2}(0) = 0,2 $ \\
c) $ x_{1}(0) = 3 $ \hspace{1cm} $ x_{2}(0) = 0 $ \\
d) $ x_{1}(0) = 0,001 $ \hspace{1cm} $ x_{2}(0) = 0,001 $\\

\subsection{Ogólny opis zagadnienia rozwiązywania układu równań różniczkowych}
Rozważane jest zagadnienie układu równań (pogrubione wartości to wektory) różniczkowych zwyczajnych pierwszego rzędu (ale mogą być one nieliniowe).
Dane jest równanie (układ rówań):\\
$ \mathbf{x'}(t) = \mathbf{f}(t,\mathbf{x}) $, przy czym \textbf{x} to szukana funkcja. Znany jest przedział, na którym szukamy \textbf{x}: $t \in [a,b]$ oraz warunki początkowe: $\mathbf{x}(a)$, . \\Wyróżnia się metody jednokrokowe, bazujące tylko na punckie otzymanym w poprzedniej iteracji, oraz metody wielokrokowe, które opierają się na większej liczbie punktów.\\

\section{Metoda RK4 ze stałym krokiem}

\subsection{Opis algorytmu}
Metody Rungego - Kutty to grupa metod jednokrokowych. Na przedziale $[t_{n}, t_{n} + h]$ obliczane są pochodne \textbf{x} (poprzez podstawienie do funkcji \textbf{f} danej równaniu), w różnych punktach w badanym przedziale, a następnie badana funkcja jest przybliżana przez pewną liniową kombinację tych pochodnych. Kompromisem pomiędzy dokładnością (rzędem metody) a nakładem obliczeń na jedną iterację jest metoda RK4 (obliczenie pochodnej w 4 punktach). Wzory opisujące jedną iterację:
$$ \mathbf{x_{n+1}} = \mathbf{x_{n}} + \frac{1}{6}h(\mathbf{k_{1}} + 2\mathbf{k_{2}} + 2\mathbf{k_{3}} + \mathbf{k_{4}})$$,
$$\mathbf{k_{1}} = \mathbf{f}(t_{n}, \mathbf{x_{n}}$$,
$$\mathbf{k_{2}} = \mathbf{f}(t_{n}+\frac{1}{2}h, \mathbf{x_{n}} + \frac{1}{2}h\mathbf{k_{1}}$$,
$$\mathbf{k_{3}} = \mathbf{f}(t_{n}+\frac{1}{2}h, \mathbf{x_{n}} + \frac{1}{2}h\mathbf{k_{2}}$$
$$\mathbf{k_{4}} = \mathbf{f}(t_{n}+h, \mathbf{x_{n}} + h\mathbf{k_{3}}$$

\subsection{Kod funkcji w środowisku MATLAB}

\end{document}


